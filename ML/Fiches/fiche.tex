\documentclass{article}
\usepackage[utf8]{inputenc}
\usepackage[a4paper, margin=2.5cm]{geometry}
\usepackage{graphicx}
\usepackage[french]{babel}

\usepackage[default,scale=0.95]{opensans}
\usepackage[T1]{fontenc}
\usepackage{amssymb} %math
\usepackage{amsmath}
\usepackage{amsthm}
\usepackage{systeme}

\usepackage{hyperref}
\hypersetup{
    colorlinks=true,
    linkcolor=blue,
    filecolor=magenta,      
    urlcolor=cyan,
    pdftitle={Overleaf Example},
    % pdfpagemode=FullScreen,
    }
\urlstyle{same} %\href{url}{Text}

\theoremstyle{plain}% default
\newtheorem{thm}{Théorème}[section]
\newtheorem{lem}[thm]{Lemme}
\newtheorem{prop}[thm]{Proposition}
\newtheorem*{cor}{Corollaire}
%\newtheorem*{KL}{Klein's Lemma}

\theoremstyle{definition}
\newtheorem{defn}{Définition}[section]
\newtheorem{exmp}{Exemple}[section]
% \newtheorem{xca}[exmp]{Exercise}

\theoremstyle{remark}
\newtheorem*{rem}{Remarque}
\newtheorem*{note}{Note}
%\newtheorem{case}{Case}



\title{Fiche ML}
\author{Charles Vin}
\date{S2-2023}

\begin{document}
\maketitle
\section{Généralité}
\begin{itemize}
    \item Fonction de perte : quantifie l'erreur associé à une décision. Erreur simple : A chaque fois qu'on se trompe, on compte 1 : 0-1 loss
    \item Risque : Proba de se tromper, $ R(y_i | x) = \sum_{j}^{} l(y_i, y_j)P(y_j | x)$
\end{itemize}

\section{Arbre de décision}

\section{Classfieur bayesien}
On a : 
\begin{itemize}
    \item $ P(y) $ fréquence des classe dans le dataset
    \item $ P(x|y) $ les points de notre jeux de donnée. Graphiquement : les points coloriés
\end{itemize}
On cherche : 
\[
    \arg \max _y P(y|x) = \arg \max _y \frac{P(x|y) P(y)}{P(x)}
.\]
$ P(x) $ difficile à calculer = répartition des points dans l'espace, dans le graph 2d non colorié. En général très petit, uniquement utile pour générer des données, pas pour faire l'argmax (aka classifier).
Classifier bayésien = le classifier qui minimise le risque = le meilleurs classifieur possible

\section{Estimation de densité}
\subsection{Par histogramme}

\begin{defn}[Estimation par histogramme]
    \begin{itemize}
        \item Cas discret : Comptage dans chaque classe puis normalisation par le nombre d'exemple $ N $ 
        \item Cas continue : Discrétisation des valeurs puis comptage et normalisation
    \end{itemize}
\end{defn}
Importance de la discrétisation : \begin{itemize}
    \item Petit $\rightarrow$ sur-apprentissage, 
    \item Trop grand $\rightarrow$ sous-apprentissage
\end{itemize}
Limite : \begin{itemize}
    \item Grande dimention $\rightarrow$ Perte de sens
    \item Effet de bord : petit changement dans les bins, gros changement d'estimation. 
\end{itemize}
$\rightarrow$ Solution : Estimation par noyaux

\subsection{Estimation de densité par noyaux}
\begin{figure}[htbp]
    \centering
    \includegraphics*[width=\textwidth]{./fig1.png}
    \caption{Intuition de l'estimation par noyaux}
    \label{intuition_noyaux}
\end{figure}
Intuition figure \ref*{intuition_noyaux} : Plutôt que de décider d'une discrétisation a priori, l'estimation est faîte en centrant une fenêtre autour du point d'intérêt $ x_0 $  (dans un espace de dimension $d$). $\rightarrow$ Problème : pas continue (si on bouge la boite et qu'un point rentre dedans, ça fait faire un saut à la fonction)

\subsubsection{Fenêtre de Parzen}
On combine la solution précédente avec une densité/noyaux. Classiquement Gaussien. pour obtenir un truc lisse et continue
\begin{defn}[Fenêtre de Parzen]
    Soit $ (x_1, \dots, x_N) \sim f $ iid 
    \[
        \hat{f}_h(x) = \frac{1}{N*h} \sum_{i=1}^{N}K (\frac{x - x_i}{h})
    .\]
    Avec $ K $ le noyaux \textbf{centrée et réduit sur $ x $ }, souvent une fonction gaussienne. Si c'est une fonction rectangle ça fonctionne aussi. Puis y'a plein d'autre noyaux possible.
\end{defn}





\end{document}