\documentclass{article}
\usepackage[utf8]{inputenc}
\usepackage[a4paper, margin=2.5cm]{geometry}
\usepackage{graphicx}
\usepackage[french]{babel}

\usepackage[default,scale=0.95]{opensans}
\usepackage[T1]{fontenc}
\usepackage{amssymb} %math
\usepackage{amsmath}
\usepackage{amsthm}
\usepackage{systeme}

\usepackage{hyperref}
\hypersetup{
    colorlinks=true,
    linkcolor=blue,
    filecolor=magenta,      
    urlcolor=cyan,
    pdftitle={Overleaf Example},
    % pdfpagemode=FullScreen,
    }
\urlstyle{same} %\href{url}{Text}

\theoremstyle{plain}% default
\newtheorem{thm}{Théorème}[section]
\newtheorem{lem}[thm]{Lemme}
\newtheorem{prop}[thm]{Proposition}
\newtheorem*{cor}{Corollaire}
%\newtheorem*{KL}{Klein’s Lemma}

\theoremstyle{definition}
\newtheorem{defn}{Définition}[section]
\newtheorem{exmp}{Exemple}[section]
% \newtheorem{xca}[exmp]{Exercise}

\theoremstyle{remark}
\newtheorem*{rem}{Remarque}
\newtheorem*{note}{Note}
%\newtheorem{case}{Case}



\title{Cours}
\author{Charles Vin}
\date{Date}

\begin{document}
\maketitle

\section{TD2}
\subsection{Exericice 1}
Diapo 3 du cours 2.
\begin{enumerate}
    \item Le score vaut 1 si le doc contient tous les termes.\begin{align*}
        RSV(q_1, d_1) &= 0 \\
        RSV(q_1, d_2) &= 1 \\
        RSV(q_1, d_3) &= 0 \\
        RSV(q_2, d_1) &=  \\
        RSV(q_2, d_2) &=  \\
        RSV(q_2, d_3) &=  \\
        RSV(q_3, d_1) &=  \\
        RSV(q_3, d_2) &=  \\
        RSV(q_3, d_3) &=  \\
    \end{align*}
    \item Cosinus similarity : $ \frac{XY}{\left\| X \right\| \left\| Y \right\| } $ 
\end{enumerate}


\section{TD4 : PageRank}
\subsection{Exercice 1}
Soit $ a_{ij} = 1 $ si lien de $ i $ vers $ j $, $ d_i = \sum_{j}^{} a_{ij} $, $ p_{ij} = \frac{a_{ij}}{d_i} $ probabilité de transition uniforme.
Score du noeud $j,  s_j = d \sum_{i}^{} p_{ij} s_i  + (1 - d)a_j$ proba que la page $ j $ soit importante. $ d $ facteur d'amortissement en général $ 0.8 $.

\textbf{Version matricielle} 
\[
    s = dsP + (1 - d)a
.\]

\begin{align*}
    A &= \begin{pmatrix}
        0 & 1 & 1 & 0 \\
        0 & 0 & 1 & 0 \\
        1 & 0 & 0 & 0 \\
        0 & 0 & 1 & 0 \\
    \end{pmatrix} \\

    P &= \begin{pmatrix}
        0 & 1 & 1/3 & 0 \\
        0 & 0 & 1/3 & 0 \\
        1 & 0 & 0   & 0 \\
        0 & 0 & 1/3 & 0 \\
    \end{pmatrix} \\
    
\end{align*}

Initialisation de $ s $ aléatoirement ça doit sommer à 1 ! 

\begin{itemize}
    \item Itération 1
    \[
        s^1 = 0.85 * 0 + 0.15 * A = \begin{pmatrix}
            0 & 0.15 & 0.15 & 0 \\
            0 & 0 & 0.15 & 0 \\
            0.15 & 0 & 0 & 0 \\
            0 & 0 & 0.15 & 0 \\
        \end{pmatrix}
    .\]
     \item Itération 2 
     \[
        s^1 = \begin{pmatrix}
            51/400 & 0 & 17/400 & 0 \\
            51/400 & 0 & 0 & 0 \\
            0 & 51/400 & 17/400 & 0 \\
            51/400 & 0 & 0 & 0 \\
        \end{pmatrix} + 0.15 * A
     .\]
     
\end{itemize}
\end{document}